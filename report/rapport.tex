% Author: Till Tantau
% Source: The PGF/TikZ manual
\documentclass[a4paper,11pt]{article}
\usepackage[utf8]{inputenc}
\usepackage{listings}
\usepackage{amsmath}    % need for subequations
\usepackage{graphicx}   % need for figures
\usepackage{verbatim}   % useful for program listings
\usepackage{color}      % use if color is used in text
\usepackage{subfigure}  % use for side-by-side figures
\usepackage{hyperref}   % use for hypertext links, including those to external documents and URLs
\usepackage{url}
\usepackage{float}
\usepackage{tikz}
\usepackage{enumitem}
\usepackage{hyperref}
\usepackage{pdfpages}
\usepackage{listings}
\usepackage{color}
\usepackage[T1]{fontenc} % use for allowing < and > in cleartext
\usepackage{fixltx2e}    % use for textsubscript

\bibliographystyle{plain}
\begin{document}
\date{1. December 2013}
\title{Bla Bla Bla\\Underblabla}
\author{Marcus Gregersen\and Martin Faartoft\and Rick Marker}
\clearpage\maketitle
\thispagestyle{empty}
\begin{abstract}
Lolleren abstract
\end{abstract}
\newpage
\setcounter{page}{1}
\section{Preface}
\section{Problem 1 - Best buddies}
%Hvad skriver vi her?

\section{Problem 2 - Popular}
\subsection{Introduction}
%RDAM
\subsection{Sequential Algorithm}
%RDAM
\subsection{Map-Reduce Algorithm}
%MLFA
Solving the \emph{Popular} problem using Map-Reduce with the specified input format, requires two rounds. The first round will re-project the original input, while the second round will count the number of unique pairings of actors.

\subsubsection{Round One}
Goal: Transform input pairs (Actor, List<Movie>) into output pairs (Movie, List<Actor>).\\

The mappers break down the (Actor, List<Movie>) pairs into a series of (Movie, Actor) pairs, one for each Movie in the input List<Movie>, essentially saying "this actor played in this movie".\\

The reducers each receive a list of (Movie, Actor) pairs (no reducer receives pairs with different keys), and builds from those a pair of shape (Movie, List<Actor>) by simply appending each actor to a list. The output of round one is thus, a pivot of the input data changing the shape from (Actor, List<Movie>) to (Movie, List<Actor>), essentially saying "this movie contained exactly these actors".

\subsubsection{Round Two}
Goal: Transform input pairs (Movie, List<Actor>) into output pairs (Actor, Count), where count is the number of unique co-actors this actor starred with in a movie.\\

The mappers break down the (Movie, List<Actor>) pairs into all possible Actor-pairs along with their inverses.

%TODO set up as nice pseudocode listing
for(actor a in actors)
	for(actor b in actors)
		if(a != b)
			emit(a, b)

This produces a number of pairs, that is quadratic in the number of actors in the movie, each capturing the information "Actor A played with Actor B in some movie)".\\

The reducers each receive a list of (Actor\textsubscript{key}, Actor\textsubscript{value}) pairs, and using a Java HashSet, they filter out any duplicate pairs, and emit the final (Actor, count) pair. These pairs show the number of unique co-actors, a given actor has had during his career.

\subsubsection{Analysis}
To gauge the efficiency of our implementation, we look at three measures. The number of rounds in the computation, the amount of work done in each mapper/reducer in the worst case, and the number of pairs emitted.\\

The algorithm runs in a constant number of rounds (2), independent of the problem size. As mentioned previously, this can be reduced to a single round, if the input has the desired structure.\\

The amount of work done in each mapper/reducer is close to balanced. Obviously, the round two reducer assigned to the most productive actor in the dataset, will do more work than the reducer assigned to an actor that only ever starred in a single movie, but in terms of the input size, these numbers are roughly equal. This is true because no single actor has played in a significant fraction of movies in the dataset.\\
If the dataset was close to being fully connected, additional tricks would be needed to balance the workload.

The number of pairs generated, is 2 for every unique pair of actors that starred in a movie together (Actor\textsubscript{A}, Actor\textsubscript{B}) and vice versa.\\
This is a large amount of pairs, that we would like to reduce if possible. We cannot see how this is possible, without relaxing the constraint that a pair of actors should be counted only once. If that constraint is lifted, the round two mapper, can simply emit (Actor, Movies.count - 1), meaning that "this actor starred with x other actors in some movie". The reducers could then simply sum all those counts, to get the total.\\
The relaxed version of round two produces drastically fewer pairs (one pair for each actor, for each movie they starred in), compared to the strict version (two pairs for each unique pair of actors per movie). By making the mapper slightly smarter, the shuffler and reducers do much less work. But the end result answers a slightly different question.\footnote{This optimization captures the same idea as the standard 'Wordcount' optimization example, where the mapper keeps a local dictionary of words and counts, instead of mindlessly emitting (word, 1) pairs.}

\subsection{Verification of results}
%MABG
\subsection{Benchmark}

\section{Appendix}

\end{document}
