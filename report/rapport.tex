% Author: Till Tantau
% Source: The PGF/TikZ manual
\documentclass[a4paper,11pt]{article}
\usepackage[utf8]{inputenc}
\usepackage{listings}
\usepackage{amsmath}    % need for subequations
\usepackage{graphicx}   % need for figures
\usepackage{verbatim}   % useful for program listings
\usepackage{color}      % use if color is used in text
\usepackage{subfigure}  % use for side-by-side figures
\usepackage{hyperref}   % use for hypertext links, including those to external documents and URLs
\usepackage{url}
\usepackage{float}
\usepackage{tikz}
\usepackage{enumitem}
\usepackage{hyperref}
\usepackage{pdfpages}
\usepackage{listings}
\usepackage{color}
\usepackage[T1]{fontenc}
\usepackage{fixltx2e}

\bibliographystyle{plain}
\begin{document}
\date{1. December 2013}
\title{Bla Bla Bla\\Underblabla}
\author{Marcus Gregersen\and Martin Faartoft\and Rick Marker}
\clearpage\maketitle
\thispagestyle{empty}
\begin{abstract}
Lolleren abstract
\end{abstract}
\newpage
\setcounter{page}{1}
\section{Preface}
\section{Problem 1 - Best buddies}
%Hvad skriver vi her?

\section{Problem 2 - Popular}
\subsection{Introduction}
%RDAM
\subsection{Sequential Algorithm}
%RDAM
\subsection{Map-Reduce Algorithm}
%MLFA
Solving the \emph{Popular} problem using Map-Reduce with the specified input format, requires two rounds. The first round will re-project the original input, while the second round will count the number of unique pairings of actors.

\subsubsection{Round One}
Goal: Transform input pairs (Actor, List<Movie>) into output pairs (Movie, List<Actor>).\\

The mappers break down the (Actor, List<Movie>) pairs into a series of (Movie, Actor) pairs, one for each Movie in the input List<Movie>, essentially saying "this actor played in this movie".\\

The reducers each receive a list of (Movie, Actor) pairs (no reducer receives pairs with different keys), and builds from those a pair of shape (Movie, List<Actor>) by simply appending each actor to a list. The output of round one is thus, a pivot of the input data changing the shape from (Actor, List<Movie>) to (Movie, List<Actor>), essentially saying "this movie contained exactly these actors".

\subsubsection{Round Two}
Goal: Transform input pairs (Movie, List<Actor>) into output pairs (Actor, Count), where count is the number of unique co-actors this actor starred with in a movie.\\

The mappers break down the (Movie, List<Actor>) pairs into all possible Actor-pairs along with their inverses.

%TODO set up as nice pseudocode listing
for(actor a in actors)
	for(actor b in actors)
		if(a != b)
			emit(a, b)

This produces an amount of pairs that is quadratic in the number of actors in the movie, each capturing the information "Actor A played with Actor B in some movie)".\\

The reducers each receive a list of (Actor\textsubscript{key}, Actor\textsubscript{value}) pairs



	hvor mange par?
	kan det gøres med færre?
		ja - HVIS vi er ligeglade med duplicates

\subsection{Verification of results}
%MABG
\subsection{Benchmark}

\section{Appendix}

\end{document}
